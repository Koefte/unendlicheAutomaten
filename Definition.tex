\documentclass[a4paper,12pt]{article}
\usepackage[utf8]{inputenc}
\usepackage{amssymb,amsmath,latexsym}
\usepackage[a4paper,top=3cm,bottom=2cm,left=3cm,right=3cm]{geometry}
\usepackage{hyperref}

\begin{document}

\section{Unendliche Automaten}

Ein unendlicher Automat wird beschrieben durch das Tupel
\[
M = (\Sigma, \varphi, A, \delta, E, r),
\]
wobei:
\begin{itemize}
    \item $\Sigma$ das Eingabealphabet ist.
    \item $\varphi$ die Menge aller Abbildungen $\mathbb{R} \to \mathbb{R}$ ist.
    \item $A$ die Menge aller Abbildungen $a : \mathbb{R} \to \{0,1\}$ ist.
    \item $\delta : A \times \Sigma \to \varphi$ die Übergangsfunktion ist.
    \item $E \subseteq A$ die Menge der Endzustände ist.
    \item $r \in \mathbb{R}$ der Anfangswert ist.
\end{itemize}

Dabei werden die Werte $\{0, 1\}$ als Wahrheitswerte interpretiert.

\subsection{Abgrenzung zum endlichen Automaten}

Ein unendlicher Automat unterscheidet sich von einem endlichen Automaten dadurch, dass er keinen endlichen Zustandsraum besitzt, sondern einen Wert aus $\mathbb{R}$ speichert.

\paragraph{Eingabeverarbeitung.}
Wie bei einem herkömmlichen endlichen Automaten wird die Eingabe vom unendlichen Automaten linear abgearbeitet, d.h. für ein Eingabewort der Länge $n$ werden die
Symbole der Eingabe in der Reihenfolge $1, 2, \dots, n$ nacheinander gelesen. Dies bedeutet nicht, dass der Automat insgesamt nur $n$ Schritte ausführen darf, da nach jedem Schritt die Rechenfunktionen $\varphi$ auf dem aktuellen Wert $r$ beliebig komplexe Berechnungen durchführen können.

\subsection{Nichtdeterministisches Beispiel}

Zur Zeit $n$ speichert der Automat eine Menge von Werten
\[
W_n \subseteq \mathbb{R}.
\]

Für ein $w \in W_n$ ist die aktive Zustandsmenge
\[
Z(w) = \{ a \in A \mid a(w) = 1 \}.
\]

$Z(w)$ ist ein Endzustand, falls
\[
Z(w) \cap E \neq \emptyset.
\]

Falls kein Endzustand erreicht ist, ergibt die Übergangsfunktion die Menge der Folgezustände:
\[
W_{n+1}(w) = \{ f(w) \mid f \in F \},
\qquad
F = \{ \delta(a,b) \mid a \in Z(w) \}.
\]

\section{Heuristische Wachstumsabschätzung}


\subsection{Wachstumsabschätzung der Zustände}
Für jedes $a \in A$ betrachten wir die Menge möglicher Folgezustände
\[
A'_{a,f}
=
\{\, a' \in A \mid a(w)=1 \Rightarrow a'(f(w)) = 1 \,\},
\quad f = \delta(a,b).
\]

Wir definieren
\[
c_a
=
\frac{1}{|\Sigma_a|}
\sum_{b \in \Sigma_a} |A'(a,\delta(a,b))|,
\]
wobei
\[
\Sigma_a = \{ b \in \Sigma \mid \delta(a,b) \neq \emptyset \}.
\]

Die Wachstumsrate des Automaten ist
\[
w = \frac{1}{|A|} \sum_{a \in A} c_a.
\]

\subsection{Wachstumsabschätzung von r}

Für jedes $b \in \Sigma$ definieren wir 
\[
F_b = \delta(A \times b)
\]
Wir konstruieren nun eine Funktion 
\[
\lambda : \mathbb{R} \to \mathbb{R}, \qquad 
x \mapsto \lambda(x),
\]
wobei
\[
\lambda_b(x) = \frac{\displaystyle \prod_{f \in F_b} f(x)}{|F_b|}.
\]

Dann definieren wir 
\[
w_r = \frac{\displaystyle \prod_{b \in \Sigma} \lambda_b}{|\Sigma|} 
\]
als Wachtumsrate von r




\section{Sprachen des Automaten}

Da jeder endliche Automat ein Spezialfall eines unendlichen Automaten ist, können unendliche Automaten reguläre Sprachen erkennen.

Weiterhin ist bekannt, dass Kellerautomaten (Pushdown-Automaten) kontextfreie Sprachen erkennen.  
Wir zeigen, dass man jeden Kellerautomaten in einen unendlichen Automaten übersetzen kann.

\subsection{Kodierung von Zustand und Keller}

Wir interpretieren
\[
\lfloor r \rfloor \text{ als Zustand}, \qquad r \bmod 1 \text{ als Kellerinhalt}.
\]

Für einen Kellerautomaten mit $n$ Zuständen definieren wir:
\[
A =
\left\{
x \mapsto
\left( \lfloor x \rfloor = i \right)
\;\middle|\;
i \in \{1,\dots,n\}
\right\}.
\]

Es bleibt, den Kellerspeicher in $(0,1)$ zu kodieren.

\subsection{Gödelisierung des Kellers}

Sei $\Gamma$ das Kelleralphabet.  
Da $\Gamma$ endlich ist, existiert eine injektive Kodierung
\[
\mathrm{code} : \Gamma \to \mathbb{N}.
\]

Für ein Wort $g = g_1 g_2 \dots g_n \in \Gamma^\*$ sei
\[
x_i = p_i^{\mathrm{code}(g_i)},
\]
wobei $(p_i)$ die Folge der Primzahlen ist.

Wir benutzen die Funktionen
\[
\psi(x) = \frac{1}{x+1},
\qquad
\psi^{-1}(y) = \frac{1}{y} - 1.
\]

Die Kellerkodierung ist
\[
\kappa(g) = \psi\left( \prod_{i=1}^n x_i \right).
\]

\subsection{Stack-Operationen}

\paragraph{Top-Operation.}
\[
\mathrm{top}(r) = \max\{ e \mid 2^e \text{ teilt } \psi^{-1}(r) \}.
\]

\paragraph{Pop-Operation.}
\[
\mathrm{pop}(r)
=
\frac{\psi^{-1}(r)}{2^{\mathrm{top}(r)}}.
\]

\paragraph{Push-Operation.}
\[
\mathrm{push}_r(g)
=
\psi\!\left(
2^{\mathrm{code}(g)} \cdot
\mathrm{shift}\bigl( \kappa^{-1}(\psi^{-1}(r \bmod 1)) \bigr)
\right),
\]
wobei
\[
\mathrm{shift}(g_1 g_2 \dots g_n)
=
\prod_{i=1}^n p_{i+1}^{\mathrm{code}(g_i)}.
\]

\section{Übersetzung der Kellerautomaten-Übergänge}

Ein Übergang des Kellerautomaten sei gegeben durch das Quintupel:
\[
(z_i, a, g_i, z_j, g_j),
\]
mit Zustand $z_i$, Eingabe $a$, Top-Symbol $g_i$, Folgezustand $z_j$ und neuem Symbol $g_j$.

Wir konstruieren dazu
\[
\hat a \in A,
\qquad
\hat a(r) = 1
\iff
\left( \lfloor r \rfloor = z_i \text{ und } \mathrm{top}(r \bmod 1) = \mathrm{code}(g_i) \right).
\]

Die Übergangsfunktion des unendlichen Automaten ist nun
\[
\delta(\hat a, a) = f,
\]
mit
\[
f(r) = z_j + \mathrm{push}_{\mathrm{pop}(r \bmod 1)}(g_j).
\]

Abschließend definieren wir den Startwert
\[
r = z_0 + \kappa(\#),
\]
wobei \# das Anfangssymbol des Kellers bezeichnet.

Damit kann zu jedem Kellerautomaten ein äquivalenter unendlicher Automat konstruiert werden.  
Daraus folgt, dass unendliche Automaten kontextfreie Sprachen entscheiden können.

\end{document}
